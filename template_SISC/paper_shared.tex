% SIAM Shared Information Template
% This is information that is shared between the main document and any
% supplement. If no supplement is required, then this information can
% be included directly in the main document.


% Packages and macros go here
\usepackage{lipsum}
\usepackage[export]{adjustbox}
\usepackage{amsfonts}
\usepackage{graphicx}
\usepackage{epstopdf}
\usepackage{algorithmic}
\usepackage{cancel}
\usepackage{caption}
\usepackage{tikz}
\usepackage{pgfplots}
\usetikzlibrary{shapes.arrows, patterns, calc}
\usepackage{tikz-3dplot}
\usepackage{bbm}
\ifpdf
  \DeclareGraphicsExtensions{.eps,.pdf,.png,.jpg}
\else
  \DeclareGraphicsExtensions{.eps}
\fi

\usepgfplotslibrary{groupplots}
\usetikzlibrary{arrows}
\usetikzlibrary{shapes}
\usetikzlibrary{decorations.text}
\usetikzlibrary{quantikz}
\usepackage{siunitx}
\usetikzlibrary{arrows.meta}


\pgfplotsset{ compat=1.18,
    standard/.style={
    scale only axis,
    width=0.5\textwidth,
    enlarge x limits=0.05,
    enlarge y limits=0.05,
    max space between ticks=40,
    every axis/.append style={font=\normalsize},
	every legend/.append style={font=\normalsize},
	every node/.append style={font=\normalsize},	
	}
}

%=============
% Hyperlink colors
%=============
% \usepackage[usenames,dvipsnames]{xcolor}
\definecolor{steelblue}{HTML}{A1BDC7}
\definecolor{orange}{HTML}{D98C21}
\definecolor{silver}{HTML}{B0ABA8}
\definecolor{rust}{HTML}{B8420F}
\definecolor{seagreen}{HTML}{2E6B69}
\definecolor{joshua}{HTML}{FBDC7F}
\definecolor{darksky}{HTML}{154c79}

\colorlet{lightsilver}{silver!30!white}
\colorlet{darkorange}{orange!85!black}
\colorlet{darksilver}{silver!85!black}
\colorlet{darksteelblue}{steelblue!85!black}
\colorlet{darkrust}{rust!85!black}
\colorlet{darkseagreen}{seagreen!85!black}

% \hypersetup{colorlinks=true,linkcolor=darkrust,citecolor=darkseagreen,urlcolor=darksilver}
\usepackage{hyperref}
\usepackage{cleveref}  % Must come AFTER hyperref
\hypersetup{colorlinks=true,linkcolor=darkrust,citecolor=darkseagreen,urlcolor=darksilver}
%=================================================
% Math macros
%=================================================

%=============
% Generalities
%=============
\usepackage{mathtools}
\usepackage{stackengine}
\usepackage{fixmath}
\usepackage{xcolor}
\usepackage{MnSymbol}


%\usepackage{was}
%\DeclareMathAlphabet{\mathbold}{OML}{cmm}{b}{it}
\mathtoolsset{centercolon}  % Makes := typeset correctly for definitions

%%% Equation numbering
%\numberwithin{equation}{section} 

%%% Annotations
\newcommand{\notate}[1]{\textcolor{red}{\textbf{[#1]}}}

%==============
% Symbols
%==============
\let\oldphi\phi
\let\oldeps\epsilon

\renewcommand{\phi}{\varphi}
\renewcommand{\epsilon}{\varepsilon}
\newcommand{\eps}{\varepsilon}

%==============
% Constants
%==============

% Set constants upright
\newcommand{\cnst}[1]{\mathrm{#1}}  
\newcommand{\econst}{\mathrm{e}}

\newcommand{\zerovct}{\vct{0}} % Zero vector
\newcommand{\Id}{\mathbf{I}} % Identity matrix
\newcommand{\onemtx}{\bm{1}}
\newcommand{\zeromtx}{\bm{0}}

%==============
% Sets
%==============
\providecommand{\mathbbm}{\mathbb} % In case we don't load bbm
\DeclareMathOperator*{\argmax}{arg\,max}
\DeclareMathOperator*{\argmin}{arg\,min}

% Reals, complex, naturals
\newcommand{\R}{\mathbbm{R}}
\newcommand{\C}{\mathbbm{C}}
\newcommand{\K}{\mathbbm{K}}
\newcommand{\N}{\mathbbm{N}}
\newcommand{\abs}[1]{\left|#1\right|}

%==============
% Probability
%==============
\newcommand{\Prob}{\operatorname{\mathbbm{P}}}
\newcommand{\Expect}{\operatorname{\mathbb{E}}}

%==============
% Vectors and matrices 
%==============
\newcommand{\vct}[1]{\mathbold{#1}}
\newcommand{\mtx}[1]{\mathbold{#1}}
\newcommand{\tp}{{T}}
\newcommand{\ad}{{*}}

\newcommand{\mrange}{\operatorname{range}}
\newcommand{\mnull}{\operatorname{null}}
\newcommand{\trace}{\operatorname{tr}}

%==============
% Differential Geometry
%==============
\newcommand{\DExp}[3]{\operatorname{Exp}_{\left(#2\right)}^{#1} \left(#3\right)}

\newcommand{\Exp}[2]{\operatorname{Exp}_{\left(#1\right)}\left(#2\right)}

\newcommand{\Tangent}[2]{T_{#1}{#2}}

\newcommand{\manif}{\mathcal{M}}
\newcommand{\ambient}{\mathcal{A}}
\newcommand{\extended}{\mathcal{E}}

%==============
% Information Geometry
%==============

\newcommand{\Diff}{\operatorname{Diff}}
\newcommand{\diff}{\Phi}
\newcommand{\deriv}[1]{#1^\prime}

\newcommand{\bpot}{\psi}
\newcommand{\emb}{\xi}

% Bregman divergence
\DeclarePairedDelimiterX{\infdivx}[2]{(}{)}{%
  #1\;\delimsize\|\;#2%
}
\newcommand{\Div}[1]{\mathbb{D}_{#1}\infdivx}

%==============
% PDEs
%==============
\newcommand{\divergence}{\operatorname{div}}
% Differential operator
\newcommand\D{\mathop{}\cnst{d}}
% variation
\newcommand{\variation}{\delta}
\newcommand{\pdedomain}{\Omega}

%==============
% Relation Symbols
%==============
\newcommand{\defeq}{\vcentcolon=}
\newcommand{\inprod}[1]{\left\langle #1 \right\rangle}

%==============
% Information geometric regularization
%==============
\newcommand{\Energy}{E}
\newcommand{\Pressure}{P}
\newcommand{\energy}{e}
\newcommand{\pressure}{p}

\newcommand{\HHop}{\mathcal{H}}
\newcommand{\HHpo}{\hat{\mathcal{H}}}

%==============
% Column and Row divergence operators
%==============
\newcommand{\cdiv}{\overset{\downarrow}{\operatorname{div}}}
\newcommand{\rdiv}{\overset{\rightarrow}{\operatorname{div}}}

%==============
% Canceling
%==============
\newcommand\Ccancel[2][black]{
    \let\OldcancelColor\CancelColor
    \renewcommand\CancelColor{\color{#1}}
    \cancel{#2}
    \renewcommand\CancelColor{\OldcancelColor}
}


%Auto-numbering
% \mathtoolsset{showonlyrefs}
\usepackage{autonum}
\graphicspath{{./figures/tikz}}
\usetikzlibrary{external}
\tikzexternalize[prefix=cache/]
%%%%%%%%%%%%%%%%%%%%%%%%%%%%%%%%%%%%%%%%%%%%%%%%%%%%%%%%%%%%%%%%%%%%%%%%%%%%%%%%
% Project-specific macro
%%%%%%%%%%%%%%%%%%%%%%%%%%%%%%%%%%%%%%%%%%%%%%%%%%%%%%%%%%%%%%%%%%%%%%%%%%%%%%%%
% Todo:
\newcommand{\todo}[1]{\textcolor{red}{\textbf{[#1]}}}
% Math 



%%%%%%%%%%%%%%%%%%%%%%%%%%%%%%%%%%%%%%%%%%%%%%%%%%%%%%%%%%%%%%%%%%%%%%%%%%%%%%%%



\ifpdf
  \DeclareGraphicsExtensions{.eps,.pdf,.png,.jpg}
\else
  \DeclareGraphicsExtensions{.eps}
\fi

% Add a serial/Oxford comma by default.
\newcommand{\creflastconjunction}{, and~}

% Used for creating new theorem and remark environments
\newsiamremark{remark}{Remark}
\newsiamremark{hypothesis}{Hypothesis}
\crefname{hypothesis}{Hypothesis}{Hypotheses}
\newsiamthm{claim}{Claim}

% Sets running headers as well as PDF title and authors
\headers{\todo{add title}}{\todo{...} and Florian Sch{\"a}fer}

% Title. If the supplement option is on, then "Supplementary Material"
% is automatically inserted before the title.
\title{Information Geometric Regularization Notes} %\thanks{Submitted to the editors DATE.}}
%\funding{This work was funded by the Fog Research Institute under contract no.~FRI-454.}

% Authors: full names plus addresses.
\author{ 
David Winters, Florian Sch{\"a}fer, and Qi Tang}

% \thanks{Georgia Tech  (\email{fts@gatech.edu}).}

\usepackage{amsopn}
\DeclareMathOperator{\diag}{diag}


%%% Local Variables: 
%%% mode:latex
%%% TeX-master: "infburgers"
%%% End: 
