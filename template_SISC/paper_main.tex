% SIAM Article Template
\documentclass[review,hidelinks,onefignum,onetabnum]{siamart220329}

%\usepackage{amsmath,amsfonts,amsthm,fullpage}
%\usepackage{mymath}
%\usepackage{algorithm}
%\usepackage{algorithmic}
%\usepackage{graphicx}
%\usepackage{bbm}



% \documentclass[onefignum,onetabnum]{siamart220329}
% \allowdisplaybreaks

% Information that is shared between the article and the supplement
% (title and author information, macros, packages, etc.) goes into
% ex_shared.tex. If there is no supplement, this file can be included
% directly.

% SIAM Shared Information Template
% This is information that is shared between the main document and any
% supplement. If no supplement is required, then this information can
% be included directly in the main document.


% Packages and macros go here
\usepackage{lipsum}
\usepackage[export]{adjustbox}
\usepackage{amsfonts}
\usepackage{graphicx}
\usepackage{epstopdf}
\usepackage{algorithmic}
\usepackage{cancel}
\usepackage{caption}
\usepackage{tikz}
\usepackage{pgfplots}
\usetikzlibrary{shapes.arrows, patterns, calc}
\usepackage{tikz-3dplot}
\usepackage{bbm}
\ifpdf
  \DeclareGraphicsExtensions{.eps,.pdf,.png,.jpg}
\else
  \DeclareGraphicsExtensions{.eps}
\fi

\usepgfplotslibrary{groupplots}
\usetikzlibrary{arrows}
\usetikzlibrary{shapes}
\usetikzlibrary{decorations.text}
\usetikzlibrary{quantikz}
\usepackage{siunitx}
\usetikzlibrary{arrows.meta}


\pgfplotsset{ compat=1.18,
    standard/.style={
    scale only axis,
    width=0.5\textwidth,
    enlarge x limits=0.05,
    enlarge y limits=0.05,
    max space between ticks=40,
    every axis/.append style={font=\normalsize},
	every legend/.append style={font=\normalsize},
	every node/.append style={font=\normalsize},	
	}
}

%=============
% Hyperlink colors
%=============
% \usepackage[usenames,dvipsnames]{xcolor}
\definecolor{steelblue}{HTML}{A1BDC7}
\definecolor{orange}{HTML}{D98C21}
\definecolor{silver}{HTML}{B0ABA8}
\definecolor{rust}{HTML}{B8420F}
\definecolor{seagreen}{HTML}{2E6B69}
\definecolor{joshua}{HTML}{FBDC7F}
\definecolor{darksky}{HTML}{154c79}

\colorlet{lightsilver}{silver!30!white}
\colorlet{darkorange}{orange!85!black}
\colorlet{darksilver}{silver!85!black}
\colorlet{darksteelblue}{steelblue!85!black}
\colorlet{darkrust}{rust!85!black}
\colorlet{darkseagreen}{seagreen!85!black}

% \hypersetup{colorlinks=true,linkcolor=darkrust,citecolor=darkseagreen,urlcolor=darksilver}
\usepackage{hyperref}
\usepackage{cleveref}  % Must come AFTER hyperref
\hypersetup{colorlinks=true,linkcolor=darkrust,citecolor=darkseagreen,urlcolor=darksilver}
%=================================================
% Math macros
%=================================================

%=============
% Generalities
%=============
\usepackage{mathtools}
\usepackage{stackengine}
\usepackage{fixmath}
\usepackage{xcolor}
\usepackage{MnSymbol}


%\usepackage{was}
%\DeclareMathAlphabet{\mathbold}{OML}{cmm}{b}{it}
\mathtoolsset{centercolon}  % Makes := typeset correctly for definitions

%%% Equation numbering
%\numberwithin{equation}{section} 

%%% Annotations
\newcommand{\notate}[1]{\textcolor{red}{\textbf{[#1]}}}

%==============
% Symbols
%==============
\let\oldphi\phi
\let\oldeps\epsilon

\renewcommand{\phi}{\varphi}
\renewcommand{\epsilon}{\varepsilon}
\newcommand{\eps}{\varepsilon}

%==============
% Constants
%==============

% Set constants upright
\newcommand{\cnst}[1]{\mathrm{#1}}  
\newcommand{\econst}{\mathrm{e}}

\newcommand{\zerovct}{\vct{0}} % Zero vector
\newcommand{\Id}{\mathbf{I}} % Identity matrix
\newcommand{\onemtx}{\bm{1}}
\newcommand{\zeromtx}{\bm{0}}

%==============
% Sets
%==============
\providecommand{\mathbbm}{\mathbb} % In case we don't load bbm
\DeclareMathOperator*{\argmax}{arg\,max}
\DeclareMathOperator*{\argmin}{arg\,min}

% Reals, complex, naturals
\newcommand{\R}{\mathbbm{R}}
\newcommand{\C}{\mathbbm{C}}
\newcommand{\K}{\mathbbm{K}}
\newcommand{\N}{\mathbbm{N}}
\newcommand{\abs}[1]{\left|#1\right|}

%==============
% Probability
%==============
\newcommand{\Prob}{\operatorname{\mathbbm{P}}}
\newcommand{\Expect}{\operatorname{\mathbb{E}}}

%==============
% Vectors and matrices 
%==============
\newcommand{\vct}[1]{\mathbold{#1}}
\newcommand{\mtx}[1]{\mathbold{#1}}
\newcommand{\tp}{{T}}
\newcommand{\ad}{{*}}

\newcommand{\mrange}{\operatorname{range}}
\newcommand{\mnull}{\operatorname{null}}
\newcommand{\trace}{\operatorname{tr}}

%==============
% Differential Geometry
%==============
\newcommand{\DExp}[3]{\operatorname{Exp}_{\left(#2\right)}^{#1} \left(#3\right)}

\newcommand{\Exp}[2]{\operatorname{Exp}_{\left(#1\right)}\left(#2\right)}

\newcommand{\Tangent}[2]{T_{#1}{#2}}

\newcommand{\manif}{\mathcal{M}}
\newcommand{\ambient}{\mathcal{A}}
\newcommand{\extended}{\mathcal{E}}

%==============
% Information Geometry
%==============

\newcommand{\Diff}{\operatorname{Diff}}
\newcommand{\diff}{\Phi}
\newcommand{\deriv}[1]{#1^\prime}

\newcommand{\bpot}{\psi}
\newcommand{\emb}{\xi}

% Bregman divergence
\DeclarePairedDelimiterX{\infdivx}[2]{(}{)}{%
  #1\;\delimsize\|\;#2%
}
\newcommand{\Div}[1]{\mathbb{D}_{#1}\infdivx}

%==============
% PDEs
%==============
\newcommand{\divergence}{\operatorname{div}}
% Differential operator
\newcommand\D{\mathop{}\cnst{d}}
% variation
\newcommand{\variation}{\delta}
\newcommand{\pdedomain}{\Omega}

%==============
% Relation Symbols
%==============
\newcommand{\defeq}{\vcentcolon=}
\newcommand{\inprod}[1]{\left\langle #1 \right\rangle}

%==============
% Information geometric regularization
%==============
\newcommand{\Energy}{E}
\newcommand{\Pressure}{P}
\newcommand{\energy}{e}
\newcommand{\pressure}{p}

\newcommand{\HHop}{\mathcal{H}}
\newcommand{\HHpo}{\hat{\mathcal{H}}}

%==============
% Column and Row divergence operators
%==============
\newcommand{\cdiv}{\overset{\downarrow}{\operatorname{div}}}
\newcommand{\rdiv}{\overset{\rightarrow}{\operatorname{div}}}

%==============
% Canceling
%==============
\newcommand\Ccancel[2][black]{
    \let\OldcancelColor\CancelColor
    \renewcommand\CancelColor{\color{#1}}
    \cancel{#2}
    \renewcommand\CancelColor{\OldcancelColor}
}


%Auto-numbering
% \mathtoolsset{showonlyrefs}
\usepackage{autonum}
\graphicspath{{./figures/tikz}}
\usetikzlibrary{external}
\tikzexternalize[prefix=cache/]
%%%%%%%%%%%%%%%%%%%%%%%%%%%%%%%%%%%%%%%%%%%%%%%%%%%%%%%%%%%%%%%%%%%%%%%%%%%%%%%%
% Project-specific macro
%%%%%%%%%%%%%%%%%%%%%%%%%%%%%%%%%%%%%%%%%%%%%%%%%%%%%%%%%%%%%%%%%%%%%%%%%%%%%%%%
% Todo:
\newcommand{\todo}[1]{\textcolor{red}{\textbf{[#1]}}}
% Math 



%%%%%%%%%%%%%%%%%%%%%%%%%%%%%%%%%%%%%%%%%%%%%%%%%%%%%%%%%%%%%%%%%%%%%%%%%%%%%%%%



\ifpdf
  \DeclareGraphicsExtensions{.eps,.pdf,.png,.jpg}
\else
  \DeclareGraphicsExtensions{.eps}
\fi

% Add a serial/Oxford comma by default.
\newcommand{\creflastconjunction}{, and~}

% Used for creating new theorem and remark environments
\newsiamremark{remark}{Remark}
\newsiamremark{hypothesis}{Hypothesis}
\crefname{hypothesis}{Hypothesis}{Hypotheses}
\newsiamthm{claim}{Claim}

% Sets running headers as well as PDF title and authors
\headers{\todo{add title}}{\todo{...} and Florian Sch{\"a}fer}

% Title. If the supplement option is on, then "Supplementary Material"
% is automatically inserted before the title.
\title{Information Geometric Regularization Notes} %\thanks{Submitted to the editors DATE.}}
%\funding{This work was funded by the Fog Research Institute under contract no.~FRI-454.}

% Authors: full names plus addresses.
\author{ 
David Winters, Florian Sch{\"a}fer, and Qi Tang}

% \thanks{Georgia Tech  (\email{fts@gatech.edu}).}

\usepackage{amsopn}
\DeclareMathOperator{\diag}{diag}


%%% Local Variables: 
%%% mode:latex
%%% TeX-master: "infburgers"
%%% End: 


% Optional PDF information
\ifpdf
\hypersetup{
  pdftitle={Information Geometric Regularization Notes},
  pdfauthor={David Winters, Florian Sch{\"a}fer, and Qi Tang}
}
\fi
% The next statement enables references to information in the
% supplement. See the xr-hyperref package for details.

% For now, disable the supplement
% \externaldocument[][nocite]{ex_supplement}

% FundRef data to be entered by SIAM
%<funding-group specific-use="FundRef">
%<award-group>
%<funding-source>
%<named-content content-type="funder-name"> 
%</named-content> 
%<named-content content-type="funder-identifier"> 
%</named-content>
%</funding-source>
%<award-id> </award-id>
%</award-group>
%</funding-group>

\begin{document}

\maketitle

% REQUIRED
\begin{abstract}
  \todo{ADD!}
\end{abstract}

% REQUIRED
\begin{keywords}
  \todo{ADD!}
\end{keywords}

% REQUIRED
\begin{MSCcodes}
  \todo{ADD!}
\end{MSCcodes}

\section{Introduction}
\cite{cao2024information}

\section{Problem Set Up}

The Barotropic Euler equation describes an iviscid and barotropic gas.

\[\partial_t\binom{\rho \boldsymbol{u}}{\rho}+\overrightarrow{\operatorname{div}}\binom{\rho \boldsymbol{u} \otimes \boldsymbol{u}+P(\rho) \mathbf{I}}{\rho \boldsymbol{u}}=\binom{\boldsymbol{f}}{0} \]

Unfortunately, this equation sometimes produces shocks as is most easily seen in the case of the Burgers equation having $P, f \equiv 0$. Solutions to this equation can be thought of as \textbf{dual geodesics} on the (infinite-dimensional) diffeomorphism manifold, and a shock is achieved upon reaching the boundary of this manifold. To fix this, we will modify the geometry of the manifold so that dual geodesics are kept away from the boundary using information-geometric forces. The case of an Eulerian description of Euler's equations has been solved for in [1] and is given below:


\[\left\{\begin{array}{l}\partial_t\binom{\rho \boldsymbol{u}}{\rho}+\overrightarrow{\operatorname{div}}\binom{\rho \boldsymbol{u} \otimes \boldsymbol{u}+(P(\rho)+\Sigma) \mathbf{I}}{\rho \boldsymbol{u}}=\binom{\boldsymbol{f}}{0} \\ \rho^{-1} \Sigma-\alpha \overrightarrow{\operatorname{div}}\left(\rho^{-1} \nabla \Sigma\right)=\alpha\left(\operatorname{tr}([\mathrm{D} \boldsymbol{u}])^2+\operatorname{tr}\left([\mathrm{D} \boldsymbol{u}]^2\right)\right) .\end{array}\right.\]

The term $\alpha$ controls the magnitude of the IG force, and in the limit $\alpha \rightarrow 0$, we expect to recover the solution to the original PDE. This is an alternative method to constructing solutions to the standard method of adding a vanishing viscosity term.


\section{Burgers' Equation}

The article continues by working the case of Burgers equation in detail:

\[ \partial_t u + u \, \partial_x u = \partial_t +\frac{1}{2} \partial_x u^2 = 0. \]

This equation can be solved by the method of characteristics until the time at which two characteristics intersect. At this point a shock forms and a differentiable solution no longer exists. When doing this by hand, we may create a weak solution by merging the two characteristics into a single line.

If we are evolving two characteristics on a computer, we may use interior point methods to prevent them from intersecting. This is done by adding a barrier term, $-\log(\det)$, that makes each characteristic approach the average of the two lines asymptotically. So, while we start with characteristic lines for the burgers' equation, when two lines approach one another, they asymptotically approach the average of the two slopes. To improve the approximation of a weak solution, we can rerun the simulation with a weaker barrier term and thus the two characteristics approach more quickly.

In the context of information geometry, the negative log barrier function produces a dually flat (information) geometry. That is, the barrier function $\psi = \frac{1}{2} x^2 - \alpha \log(x')$ induces a dual set of coordinates $\eta = \nabla \psi$, whose inverse exists by the strict convexity of $\psi$. This also induces a Riemannian metric via its Hessian. When we say that the Burgers' equation has a dually flat information geometry, we mean its solutions are straight lines in the coordinates of $\eta$. We may map back to coordinates in terms of $x$ via $x(t) = \nabla \psi^{-1} \eta(t) $. By adding this log term, the ``straight lines'' avoid the boundary and allow the simulation to proceed without shocks. There are some details I am missing here such as the coordinates $(\overline{\Phi}, \Phi')$ to define $\psi$, but I will instead jump to the infinite dimensional case.

Instead of considering a finite number of geodesics, we can imagine that the solution $\Phi_t$ is instead a diffeomorphism which tells us the net deformation of the initial data at time zero. As the initial data evolves, we record this as a new diffeomorphism which evolves differentiably in time. The correct notion of coordinates are 

\[ \overline{\Phi} = \int_\mathbb{R} \Phi(x) - x \; dx \quad \textrm{and} \quad \Phi' = \partial_x \Phi.  \]

So, $\overline{\Phi}$ is a one-dimensional coordinate system in the space of diffeomorphisms, while $\Phi'$ is function-valued (infinite collection of coordinates). Then the barrier in the infinite dimensional setting is:

\[ \psi(\Phi) = \psi_E(\Phi) - \alpha \int_\mathbb{R} \log (\partial_x \Phi ) \; dx \quad\quad \psi_E(\Phi) = \frac{1}{2} \| \Phi(\cdot) - ( \cdot )   \|_{L^2}^2. \]

If we extend from Burgers to Euler where $\Phi$ takes values in $\mathbb{R}^n$:

\[ \hat{\psi}(\Phi) = \psi(\Phi , \, \det D \Phi) :=  \psi_E(\Phi) - \alpha \int_{\mathbb{R}^d} \log (\det D \Phi ) \; dx.  \]

Finally, we want to be able to recover the PDE from the deformation maps $\Phi(x)$ (change back to Eulerian coordinates). This is done via:

\[ u(\Phi(x)) = \dot{\Phi}(x)  \; \textrm{ and } \; \rho(\Phi(x)) = (\det (D\Phi))^{-1}  \]
\[\implies \; \int_{\mathbb{R}^d} \log (\det D \Phi ) \; dx = \int_{\mathbb{R}^d} \log (\rho ) \rho \; dx.\]


\section{Derivation for Flat Geometry}

They derive in $\mathbb{R}^d$ for a general barrier $\psi$. Ideas:

\begin{itemize}
    \item We can extend a tangent vectors beginning at a particular point into a ``line'' via the exponential map given by the Riemannian metric.
    \item The dual exponential map maps a tangent vector to the dual space direction, extends this to a path via the exponential map, and changes back coordinates $\eta \rightarrow x$.
    \item These geodesics tell us how things will evolve in the absence of forces.
\end{itemize}

With forces, we have a dual Newton's Second Law (derived later) for force $K$:

\[\ddot{\Phi}_t+\left[\mathrm{D}^2 \psi\left(\Phi_t\right)\right]^{-1}\left[\mathrm{D}^3 \psi\left(\Phi_t\right)\right]\left(\dot{\Phi}_t, \dot{\Phi}_t\right)=K\left(t, \Phi_t, \dot{\Phi}_t\right).\]

In the univariate case, the barrier function is:

\[\psi(\Phi)=\frac{1}{2} \int_{\mathbb{R}}|\Phi(x)-x|^2 \mathrm{~d} x+\alpha \int_{\mathbb{R}}-\log \partial_x \Phi(x) \mathrm{d} x.\]

The article goes on to explicitly solve for Newton's Second Law in terms of the barrier function given above. Because $\psi$ is a functional, to calculate its derivatives, we need to take variations of the functional and these are derived in 3.2.2. The conclusion is the equation:

\[\left((\cdot)-\alpha \partial_x\left(\left[\partial_x \Phi_t\right]^{-2}\left[\partial_x(\cdot)\right]\right)\right)\left(\ddot{\Phi}_t-K\left(t, \Phi_t, \dot{\Phi}_t\right)\right)=-2 \alpha \partial_x\left(\left[\partial_x \Phi_t\right]^{-3}\left[\partial_x \dot{\Phi}_t\right]^2\right)\]

Next, they derive the equation in Eulerian coordinates using the equations given at the end of section 2. The final part of section 3 is to derive a new conservation law.


\section*{Questions to Answer:}

\noindent 1. How did we derive Newton's Second Law?

We will perform the case of no external forces. Using the equation for the dual geodesic:

\[ \Phi_t = Exp_{(\Phi_0)}^\psi (t \dot{\Phi}_0) = \nabla \psi^{-1} \circ (\nabla \psi(\Phi_0) + t [D^2 \psi(\Phi_0)] \dot{\Phi}_0   )  \]

In which we can clearly see that we are transforming a straight line in the dual space. Next, we take a derivative:

\[ \dot{\Phi}_t = \frac{d}{dt} \nabla \psi^{-1} \circ (\nabla \psi(\Phi_0) + t [D^2 \psi(\Phi_0)] \dot{\Phi}_0   ) = (\psi^{-1})'(\nabla \psi(\Phi_0) + t [D^2 \psi(\Phi_0)] \dot{\Phi}_0  ) \cdot [D^2 \psi(\Phi_0)] \dot{\Phi}_0 \]
\[ = D [D\psi^{-1}](\nabla \psi(\Phi_t)  ) \cdot [D^2 \psi(\Phi_0)] \dot{\Phi}_0 = [D^2\psi(\Phi_t )]^{-1} \cdot [D^2 \psi(\Phi_0)] \dot{\Phi}_0 \]

because

\[\nabla(\nabla \psi)^{-1}(y)=\left(\nabla^2 \psi\left((\nabla \psi)^{-1}(y)\right)\right)^{-1}.\]

Similarly, we can take the second derivative:

\[\ddot{\Phi}_t = \frac{d}{dt} [D^2\psi(\Phi_t )]^{-1} \cdot [D^2 \psi(\Phi_0)] \dot{\Phi}_0 = - [D^2\psi(\Phi_t )]^{-1} \frac{d[D^2\psi(\Phi_t )]}{dt} [D^2\psi(\Phi_t )]^{-1}  \cdot [D^2 \psi(\Phi_0)] \dot{\Phi}_0   \]
\[ = - [D^2\psi(\Phi_t )]^{-1} [D^3\psi(\Phi_t )](\dot{\Phi}_t) [D^2\psi(\Phi_t )]^{-1}  \cdot [D^2 \psi(\Phi_0)] \dot{\Phi}_0  = - [D^2\psi(\Phi_t )]^{-1} [D^3\psi(\Phi_t )](\dot{\Phi}_t, \dot{\Phi}_t).\]

Here we used the fact that:

\[ \frac{d}{dt} [D^2\psi(\Phi_t )][D^2\psi(\Phi_t )]^{-1} = \frac{d}{dt} I = 0 = [D^2\psi(\Phi_t )] \frac{d [D^2\psi(\Phi_t )]^{-1}}{dt} + \frac{d[D^2\psi(\Phi_t )]}{dt} [D^2\psi(\Phi_t )]^{-1}.  \]

Thus we have derived Newton's Second Law:

\[ \ddot{\Phi}_t + [D^2\psi(\Phi_t )]^{-1} [D^3\psi(\Phi_t )](\dot{\Phi}_t, \dot{\Phi}_t) = 0.   \]

\medskip\medskip

\noindent 2. How to derive the Euler-Lagrange equations? This method is ultimately not what we are looking for, but this shows an interesting derivation. 

\medskip

Here, we will be using the Lagrangian $\mathcal{L} = \frac{1}{2} \dot{\Phi}^T [D^2 \psi(\Phi)] \dot{\Phi}  $. The Euler-Lagrange equations:

\[ \frac{d}{dt} \frac{\mathcal{L}}{\partial \dot{\Phi}} = \frac{\partial \mathcal{L}}{\partial \Phi}  \quad \textrm{ becomes } \quad \frac{d}{dt} [D^2 \psi(\Phi)] \dot{\Phi}  = \frac{1}{2} [D^3 \psi](\dot{\Phi}, \, \dot{\Phi}, \, \cdot) \]

or alternatively:

\[ [D^3 \psi](\dot{\Phi}, \, \dot{\Phi}, \, \cdot) + [D^2 \psi(\Phi) ] \ddot{\Phi} = \frac{1}{2} [D^3 \psi](\dot{\Phi}, \, \dot{\Phi}, \, \cdot) \quad \implies \quad  \ddot{\Phi} + \frac{1}{2} [D^2\psi(\Phi )]^{-1} [D^3 \psi](\dot{\Phi}, \, \dot{\Phi}, \, \cdot) = 0. \]

So, if we have an external force $F(\Phi, \, \dot{\Phi})$, the equation becomes:

\[ \ddot{\Phi} + \frac{1}{2} [D^2\psi(\Phi )]^{-1} [D^3 \psi](\dot{\Phi}, \, \dot{\Phi}, \, \cdot) = F(\Phi, \, \dot{\Phi}).  \]

Instead, we would like to do the same for the dual equations.

\medskip\medskip

\noindent 3. How exactly do we parallel transport on a submanifold?




\section{Multivariate Case}


\section{Embedding onto a Submanifold}

In the general ambient space of diffeomorphisms $\mathcal{E}$, $\psi$ is convex, but it is not when we restrict to the submanifold $\mathcal{M}$ consisting of $(\Phi, \, D\Phi)$. To fix this, we will perform parallel transport in $\mathcal{E}$ via $\psi$, and then restrict our path to live in $\mathcal{M}$. This means if we do things with discrete time, we draw a part of the geodesic and project the endpoint back onto the submanifold. This projection $\pi$ induces a projection of the tangent spaces, which allows us to transport tangent vectors too. The article explicitly calculates this parallel transport of the tangent vector along $\mathcal{M}$ as the base point $\Phi$ evolves to $\Phi + \varepsilon U$. If we differentiate in $\varepsilon$ and set $\varepsilon = 0$, they derive the piece of the affine connection that describes the induced connection on $\mathcal{M}$.

\[\begin{aligned} & \Gamma_{\Phi}^{\psi, \xi}(U, V)=\left([\mathrm{D} \xi(\Phi)]^*\left[\mathrm{D}^2 \psi(\xi(\Phi))\right][\mathrm{D} \xi(\Phi)]\right)^{-1}[\mathrm{D} \xi(\Phi)]^*\left[\mathrm{D}^2 \psi(\xi(\Phi))\right] \\ & \left(\left[\mathrm{D}^2 \xi(\Phi)\right](U, V)+\left[\mathrm{D}^2 \psi(\xi(\Phi))\right]^{-1}\left[\mathrm{D}^3 \psi(\xi(\Phi))\right]([\mathrm{D} \xi(\Phi)] U,[\mathrm{D} \xi(\Phi)] V)\right) .\end{aligned} \]

So to calculate the equation of internal motion, we calculate $\left[\mathrm{D}^2 \xi(\Phi)\right]$ and $[\mathrm{D} \xi(\Phi)]^*$. After some work, we have the equation of motion involving only derivatives of $\Phi$: $D\Phi$, $\dot{\Phi}$, and $\ddot{\Phi}$.

\[ \ddot{\Phi}-\alpha \operatorname{div}\left(\textrm{ tr}\left([\mathrm{D} \Phi]^{-1}[\mathrm{D} \ddot{\Phi}]\right)[\mathrm{D} \Phi]^{-1}\right) \] 

\[= -\alpha \textrm{ div}\left( \textrm{ tr}([D \Phi]^{-1} [D \dot{\Phi}])^2 [D \Phi]^{-1}\right) - \alpha \textrm{ div}(\textrm{ tr}^2([D \Phi]^{-1} [D \dot{\Phi}]) [D\Phi]^{-1} )\]

Next, they convert the equations of motion to Eulerian by using:

\[u(\Phi(x)) = \dot{\Phi}(x)  \; \textrm{ and } \; \rho(\Phi(x)) = (\det (D\Phi))^{-1}\]

We remove the $\Phi$ dependence and get an equivalent equation only involving 

\[ \partial_t(\rho \boldsymbol{u})+\overrightarrow{\operatorname{div}}(\rho \boldsymbol{u} \otimes \boldsymbol{u}) \] 

\[=-\alpha \rho((\cdot) \rho-\alpha \overrightarrow{\operatorname{div}}(\rho \operatorname{tr}(\mathrm{D}(\cdot)) \mathbf{I}))^{-1} \overrightarrow{\operatorname{div}}\left(\operatorname{tr}\left([\mathrm{D} \boldsymbol{u}]^2\right) \rho \mathbf{I}+\operatorname{tr}^2([\mathrm{D} \boldsymbol{u}]) \rho \mathbf{I}\right)\]

Finally, we put the equation in conservation form. As in section 4.2.4, we multiply by $\rho$ and use the equation for conservation of mass. We will pull the divergence and derivatives with $\rho$ through the inverse to obtain a final equation:


$$
\left\{\begin{array}{l}
\partial_t\binom{\rho \boldsymbol{u}}{\rho}+\overrightarrow{\operatorname{div}}\binom{\rho \boldsymbol{u} \otimes \boldsymbol{u}+(P(\rho)+\Sigma) \mathbf{I}}{\rho \boldsymbol{u}}=\binom{\boldsymbol{f}}{0} \\
\rho^{-1} \Sigma-\alpha \operatorname{div}\left(\rho^{-1} \nabla \Sigma\right)=\alpha\left(\operatorname{tr}^2([\mathrm{D} \boldsymbol{u}])+\operatorname{tr}\left([\mathrm{D} \boldsymbol{u}]^2\right)\right) .
\end{array}\right.
$$


\section{Lagrangian Conservation}


\[ \rho_0(X) \partial_t^2 \varphi(t, X)=-\partial_X\left(p\left(\frac{\rho_0(X)}{\partial_X \varphi(t, X)}\right)\right) \]

The momentum equation in Lagrangian coordinates is:

\[ \rho_0 \ddot{\Phi} = \partial_x P + \rho_0 f \]

We begin with the equation (4.4). To turn this into a conservation law like the equation above, we multiply by $\rho$

\[\left((\cdot)-\alpha \partial_x\left(\left[\partial_x \Phi_t\right]^{-2}\left[\partial_x(\cdot)\right]\right)\right)\left(\ddot{\Phi}_t-K\left(t, \Phi_t, \dot{\Phi}_t\right)\right)=-2 \alpha \partial_x\left(\left[\partial_x \Phi_t\right]^{-3}\left[\partial_x \dot{\Phi}_t\right]^2\right)\]

\newpage

\section{Lagrangian Conservation Attempt 1}

We begin with the 1-D conservation equation given in Eulerian coordinates. This is equation (4.5) in the article.

\[\begin{cases}\partial_t(\rho u)+\partial_x\left(\rho u^2+P(\rho)+\Sigma\right) & =f \\ \partial_t \rho+\partial_x(\rho u) & =0 \\ \Sigma \rho^{-1}-\alpha \partial_x\left(\rho^{-1} \partial_x \Sigma\right) & =2 \alpha\left[\partial_x u\right]^2 .\end{cases}\]

Let's first recall the conservation of mass in Lagrangian form. The textbook by Stuart uses the notation,

\[ F(X,t) = \partial_x \Phi(X,t) \quad\quad \rho_m(X,t) (\det F(X,t)) = \rho_0(X)  \]

However, we use that $u$ and $\Phi$ are related via

\[u(\Phi(x)) = \dot{\Phi}(x)  \; \textrm{ and } \; \rho(\Phi(x)) = (\det (D\Phi))^{-1}\]
\[\ddot{\Phi}_t(x)=\left[\partial_x u\left(\Phi_t(x)\right)\right] u\left(\Phi_t(x)\right)+\partial_t u\left(\Phi_t(x)\right)\]

and our goal is to get something approximating one of the following momentum balance equations:

\[ \rho_0 \ddot{\Phi} = div(P) + \rho_0 b = f - \partial_X \, p\left(\frac{\rho_0}{\partial_X \Phi} \right).    \]

\par\noindent\rule{\textwidth}{0.4pt}

Working with the first term in the conservation law,

\[ \partial_t (\rho u) = \partial_t (\dot{\Phi} (\det (D\Phi))^{-1} ) = \ddot{\Phi} (\det (D\Phi))^{-1} + \dot{\Phi} \partial_t (\det (D\Phi))^{-1} \]
\[= \ddot{\Phi} (\det (D\Phi))^{-1} - \dot{\Phi} (\det (D\Phi))^{-2}\partial_t \det (D\Phi)\]

By the Jacobi Formula,

\[\partial_t \det (D\Phi) = \det (D\Phi)  \; tr\left((D\Phi)^{-1} \frac{d D\Phi}{dt}\right) \]

so we have the first term in the eulerian coordinate conservation law becomes:

\[\partial_t (\rho u)= \ddot{\Phi} (\det (D\Phi))^{-1} - \dot{\Phi} (\det (D\Phi))^{-1} \, tr\left((D\Phi)^{-1} \frac{d D\Phi}{dt}\right)\]

Or, if we want to simplify to something in 1-D, we obtain:

\[\partial_t (\rho u)= \frac{\ddot{\Phi}}{\partial_x\Phi} -  \frac{\dot{\Phi} \partial_x \dot{\Phi}}{(\partial_x\Phi)^2} = \frac{\ddot{\Phi}}{\partial_x\Phi} -  \frac{ \partial_x (\dot{\Phi}^2)}{2(\partial_x\Phi)^2}\]


Next, lets simplify some of the second term:

\[ \partial_x(\rho u^2) = \partial_x(\dot{\Phi}^2 (\det (D\Phi))^{-1}) = \partial_x\left( \frac{\dot{\Phi}^2}{\partial_x \Phi}   \right)  \]


Next, we can use that

\[ K\left(t, \Phi_t, \dot{\Phi}_t\right) \circ \Phi_t^{-1}=\partial_x P(\rho)-f \]

Finally, we need to express the equation for $\Sigma$ in Lagrangian coordinates. In Eulerian coordinates, we had

\[\Sigma \rho^{-1}-\alpha \partial_x\left(\rho^{-1} \partial_x \Sigma\right) \quad=2 \alpha\left[\partial_x u\right]^2\]

which in Lagrangian coordinates becomes:

\[ \Sigma \det (D\Phi) - \alpha \partial_x \left( \det (D\Phi) \partial_x \Sigma    \right) = 2 \alpha [\partial_x \dot{\Phi}]^2. \]

If we put our ``simplified'' parts together:

\[ \frac{\ddot{\Phi}}{\partial_x\Phi} -  \frac{ \partial_x (\dot{\Phi}^2)}{2(\partial_x\Phi)^2}+\partial_x\left( \frac{\dot{\Phi}^2}{\partial_x \Phi} + \Sigma  \right)  = f - \partial_x P\left( \frac{1}{\partial_x \Phi} \right)  \]


Then to put all three equations together:

\[
\begin{cases}\frac{\ddot{\Phi}}{\partial_x\Phi} -  \frac{ \partial_x (\dot{\Phi}^2)}{2(\partial_x\Phi)^2}+\partial_x\left( \frac{\dot{\Phi}^2}{\partial_x \Phi} + \Sigma  \right) & = f - \partial_x P\left( \frac{1}{\partial_x \Phi} \right)  \\  \rho(\Phi(x)) & = 1/ \partial_x \Phi \\ 
\Sigma (\partial_x\Phi) - \alpha \partial_x \left( \partial_x \Phi \;  \partial_x \Sigma    \right) = 2 \alpha [\partial_x \dot{\Phi}]^2 .\end{cases}
\]



\newpage

\section{Lagrangian Conservation Attempt 2}

Alternatively, we can try and create a conservation law using Newton's second law already in Lagrangian coordinates.

\[ \left((\cdot)-\alpha \partial_x\left(\left[\partial_x \Phi_t\right]^{-2}\left[\partial_x(\cdot)\right]\right)\right)\left(\ddot{\Phi}_t-K\left(t, \Phi_t, \dot{\Phi}_t\right)\right)=-2 \alpha \partial_x\left(\left[\partial_x \Phi_t\right]^{-3}\left[\partial_x \dot{\Phi}_t\right]^2\right)   \]



\section{Numerical experiments}
\label{sec:numerics}


\section{Comparison, conclusion, and outlook}
\label{sec:conclusion}









\section*{Acknowledgments}


\bibliographystyle{siamplain}
\bibliography{references}

% \appendix
% \input{appendix.tex}
\end{document}
